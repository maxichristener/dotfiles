% Created 2023-08-12 sáb 12:36
% Intended LaTeX compiler: pdflatex
\documentclass[11pt]{article}
\usepackage[utf8]{inputenc}
\usepackage[T1]{fontenc}
\usepackage{graphicx}
\usepackage{longtable}
\usepackage{wrapfig}
\usepackage{rotating}
\usepackage[normalem]{ulem}
\usepackage{amsmath}
\usepackage{amssymb}
\usepackage{capt-of}
\usepackage{hyperref}
\author{Maxi}
\date{\today}
\title{Apunte sobre Git}
\hypersetup{
 pdfauthor={Maxi},
 pdftitle={Apunte sobre Git},
 pdfkeywords={},
 pdfsubject={},
 pdfcreator={Emacs 29.1 (Org mode 9.6.6)}, 
 pdflang={English}}
\begin{document}

\maketitle
\tableofcontents


\section{Configuraciones previas}
\label{sec:org1b819f4}

Antes de hacer algo, es necesario configuar Git y GitHub,
respectivamente. Básicamente, el procedimiento va a ser generar llaves
SSH y GPG para hacer commits y firmarlos, respectivamente.

\subsection{Generando llaves SSH}
\label{sec:org0ac80e1}

Generamos una llave con algoritmo ed25519 y seguimos los pasos requeridos.
\begin{verbatim}
ssh-keygen -t ed25519 -C "your\textsubscript{email}@example.com"
\end{verbatim}

Ahora, antes de generar la llave \emph{privada} se necesita comenzar con el
cliente ssh-agent de fondo
\begin{verbatim}
eval "\$(ssh-agent -s)"
\end{verbatim}

Ya podemos añadirla a nuestra carpeta
\begin{verbatim}
ssh-add \textasciitilde{}/.ssh/id\textsubscript{ed25519}
\end{verbatim}

El archivo con terminación \texttt{.pub} es la llave pública, la que se va a
añadir a GitHub.

\subsection{Registrando llaves GPG}
\label{sec:org7e4285a}

Ya existe una guía para generar y administrar llaves GPG, útiles para
firmar.

Si previamente se usaba otra llave GPG, la quitamos.
\begin{verbatim}
git config --global --unset gpg.format
\end{verbatim}

Configuramos nuestra ID primaria en git (se presenta una como ejemplo)
\begin{verbatim}
git config --global user.signingkey 3AA5C34371567BD2
\end{verbatim}

Configuramos git para que firme automáticamente todos los commits
\begin{verbatim}
git config --global commit.gpgsign true
\end{verbatim}

\section{Usando Git}
\label{sec:orgbe6762b}

\subsection{Comenzando}
\label{sec:org2f7906f}

Hacemos un init en la carpeta que estamos trabajando. Leer atentamente
el output.
\begin{verbatim}
git init
\end{verbatim}

Alternativamente, podemos hacer clonar un repositorio
\begin{verbatim}
git clone REPO
\end{verbatim}

\subsection{Actualizando los cambios realizados}
\label{sec:orgacf4a31}

Para obtener los últimos cambios realizados en nuestro repositorio,
hacemos un \texttt{git pull}. También podemos especificar a qué rama o branch
queremos realizar el pull, haciendo \texttt{git pull origin BRANCH}.

\subsection{Realizando cambios}
\label{sec:orgf36b9f8}

Una vez hechos los cambios en nuestra computadora, necesitamos
"avisarle" a Git. Si deseamos hacerlo con todo el directorio
\begin{verbatim}
git add .
\end{verbatim}

Ahora, necesitamos confirmar lo actualizado en un "commit", para
finalmente enviarlo a nuestro repositorio. Para ello
\begin{verbatim}
git commit -m "mensaje del commit"
\end{verbatim}

"Pusheamos" el "commit", es decir, enviamos los cambios confirmados a
nuestro repositorio, en la branch que estamos trabajando
\begin{verbatim}
git push origin BRANCH
\end{verbatim}

\section{Remote}
\label{sec:org5543a99}

En algún momento, es necesario conectar nuestro directorio Git con
GitHub. Remote pone en contacto a los repositorios. El comando \texttt{git
remote} nos informa el nombre que se le adjudica, el cual es siempre
por defecto "origin".

Entonces, el proceso consiste en avisarle a GitHub vía SSH que tenemos un
remoto origin añadido en nuestra computadora. 
\begin{verbatim}
git remote add origin SSH
\end{verbatim}


\section{Extras}
\label{sec:org4dc52a1}

\subsection{Eliminar algo}
\label{sec:orgb39bfeb}
Para eliminar un archivo de GitHub, hacemos
\begin{verbatim}
git rm FILE
\end{verbatim}

Luego, es necesario añadir los cambios (\texttt{git add}), hacer un commit y
pushearlo a GitHub.

\subsection{Ver cambios modificados}
\label{sec:orgbd7d737}

Para ver si hay cambios no notificados a Git, usamos
\begin{verbatim}
git status
\end{verbatim}
\end{document}